%% BioMed_Central_Tex_Template_v1.06
%%                                      %
%  bmc_article.tex            ver: 1.06 %
%                                       %

%%IMPORTANT: do not delete the first line of this template
%%It must be present to enable the BMC Submission system to
%%recognise this template!!

%%%%%%%%%%%%%%%%%%%%%%%%%%%%%%%%%%%%%%%%%
%%                                     %%
%%  LaTeX template for BioMed Central  %%
%%     journal article submissions     %%
%%                                     %%
%%          <8 June 2012>              %%
%%                                     %%
%%                                     %%
%%%%%%%%%%%%%%%%%%%%%%%%%%%%%%%%%%%%%%%%%


%%%%%%%%%%%%%%%%%%%%%%%%%%%%%%%%%%%%%%%%%%%%%%%%%%%%%%%%%%%%%%%%%%%%%
%%                                                                 %%
%% For instructions on how to fill out this Tex template           %%
%% document please refer to Readme.html and the instructions for   %%
%% authors page on the biomed central website                      %%
%% http://www.biomedcentral.com/info/authors/                      %%
%%                                                                 %%
%% Please do not use \input{...} to include other tex files.       %%
%% Submit your LaTeX manuscript as one .tex document.              %%
%%                                                                 %%
%% All additional figures and files should be attached             %%
%% separately and not embedded in the \TeX\ document itself.       %%
%%                                                                 %%
%% BioMed Central currently use the MikTex distribution of         %%
%% TeX for Windows) of TeX and LaTeX.  This is available from      %%
%% http://www.miktex.org                                           %%
%%                                                                 %%
%%%%%%%%%%%%%%%%%%%%%%%%%%%%%%%%%%%%%%%%%%%%%%%%%%%%%%%%%%%%%%%%%%%%%

%%% additional documentclass options:
%  [doublespacing]
%  [linenumbers]   - put the line numbers on margins

%%% loading packages, author definitions

\documentclass[twocolumn]{bmcart}% uncomment this for twocolumn layout and comment line below
%\documentclass{bmcart}

%%% Load packages
\usepackage{amsthm,amsmath}
\usepackage{siunitx}
\usepackage{mfirstuc}
%\RequirePackage{natbib}
\usepackage[colorinlistoftodos]{todonotes}
\RequirePackage{hyperref}
\usepackage[utf8]{inputenc} %unicode support
%\usepackage[applemac]{inputenc} %applemac support if unicode package fails
%\usepackage[latin1]{inputenc} %UNIX support if unicode package fails
\usepackage[htt]{hyphenat}

\usepackage{array}
\newcolumntype{L}[1]{>{\raggedright\let\newline\\\arraybackslash\hspace{0pt}}p{#1}}

%%%%%%%%%%%%%%%%%%%%%%%%%%%%%%%%%%%%%%%%%%%%%%%%%
%%                                             %%
%%  If you wish to display your graphics for   %%
%%  your own use using includegraphic or       %%
%%  includegraphics, then comment out the      %%
%%  following two lines of code.               %%
%%  NB: These line *must* be included when     %%
%%  submitting to BMC.                         %%
%%  All figure files must be submitted as      %%
%%  separate graphics through the BMC          %%
%%  submission process, not included in the    %%
%%  submitted article.                         %%
%%                                             %%
%%%%%%%%%%%%%%%%%%%%%%%%%%%%%%%%%%%%%%%%%%%%%%%%%


%\def\includegraphic{}
%\def\includegraphics{}

%%% Put your definitions there:
\startlocaldefs
\endlocaldefs


%%% Begin ...
\begin{document}

%%% Start of article front matter
\begin{frontmatter}

\begin{fmbox}
\dochead{Report from 2015 OHBM Hackathon (HI)}

%%%%%%%%%%%%%%%%%%%%%%%%%%%%%%%%%%%%%%%%%%%%%%
%%                                          %%
%% Enter the title of your article here     %%
%%                                          %%
%%%%%%%%%%%%%%%%%%%%%%%%%%%%%%%%%%%%%%%%%%%%%%

\title{Highly Comparable Time-Series Analysis in Nitime}
\vskip2ex
\projectURL{Project URL: \url{https://github.com/benfulcher/hctsa\_python}}

\author[
addressref={aff1},
corref={aff1},
email={ben.fulcher@monash.edu}
]{\inits{BDF} \fnm{Ben D.} \snm{Fulcher}}

%%%%%%%%%%%%%%%%%%%%%%%%%%%%%%%%%%%%%%%%%%%%%%
%%                                          %%
%% Enter the authors' addresses here        %%
%%                                          %%
%% Repeat \address commands as much as      %%
%% required.                                %%
%%                                          %%
%%%%%%%%%%%%%%%%%%%%%%%%%%%%%%%%%%%%%%%%%%%%%%

\address[id=aff1]{%
  \orgname{Monash Institute of Cognitive and Clinical Neurosciences, Monash
University},
  \city{Melbourne},
  \street{770 Blackburn Rd},
  \postcode{3168},
  %
  \cny{Australia}
}

%%%%%%%%%%%%%%%%%%%%%%%%%%%%%%%%%%%%%%%%%%%%%%
%%                                          %%
%% Enter short notes here                   %%
%%                                          %%
%% Short notes will be after addresses      %%
%% on first page.                           %%
%%                                          %%
%%%%%%%%%%%%%%%%%%%%%%%%%%%%%%%%%%%%%%%%%%%%%%

\begin{artnotes}
\end{artnotes}

%\end{fmbox}% comment this for two column layout

%%%%%%%%%%%%%%%%%%%%%%%%%%%%%%%%%%%%%%%%%%%%%%
%%                                          %%
%% The Abstract begins here                 %%
%%                                          %%
%% Please refer to the Instructions for     %%
%% authors on http://www.biomedcentral.com  %%
%% and include the section headings         %%
%% accordingly for your article type.       %%
%%                                          %%
%%%%%%%%%%%%%%%%%%%%%%%%%%%%%%%%%%%%%%%%%%%%%%

%\begin{abstractbox}

%\begin{abstract} % abstract
	
%Blank Abstract

%\end{abstract}



%%%%%%%%%%%%%%%%%%%%%%%%%%%%%%%%%%%%%%%%%%%%%%
%%                                          %%
%% The keywords begin here                  %%
%%                                          %%
%% Put each keyword in separate \kwd{}.     %%
%%                                          %%
%%%%%%%%%%%%%%%%%%%%%%%%%%%%%%%%%%%%%%%%%%%%%%

%\vskip1ex

%\projectURL{\url{https://github.com/benfulcher/hctsa\_python}}
%\projectURL{https://github.com/benfulcher/hctsa\_python}

% MSC classifications codes, if any
%\begin{keyword}[class=AMS]
%\kwd[Primary ]{}
%\kwd{}
%\kwd[; secondary ]{}
%\end{keyword}

%\end{abstractbox}
%
\end{fmbox}% uncomment this for twcolumn layout

\end{frontmatter}

%{\sffamily\bfseries\fontsize{10}{12}\selectfont Project URL: \url{https://github.com/benfulcher/hctsa\_python}}

%%% Import the body from pandoc formatted text
\section{Introduction}\label{introduction}

The aim of this project was to begin to extend an existing Matlab-based
package for implementing thousands of time-series analysis methods,
\href{https://github.com/benfulcher/hctsa}{hctsa}, to a python-based
implementation, for potential future inclusion into
\href{http://nipy.org/nitime/}{Nitime}.

Nitime is python-based package for performing time-series analysis on
neuroscience data. The highly comparative time-series analysis approach
\cite{Fulcher2013} has an associated
\href{www.comp-engine.org/timeseries}{website} and Matlab-based code
package, \emph{hctsa}, that extracts thousands of structural features
from a time series and determines which are most useful for a given
scientific task.

In order to apply highly comparative time-series analysis in the
neuroscience community, it would be desirable to implement some
time-series analysis methods into the Nitime package, or at least using
the Nitime data format. This would facilitate not only their use by the
neuroscience community, but also their maintenance and development
within an open source framework.

\section{Approach}\label{approach}

An illustration of the approach is shown in Fig. \ref{centfig}. Each
time series is converted to a vector of thousands of informative
features using the \emph{hctsa} package, and then machine learning
methods are used to determine the most useful features. In this project,
we wanted to demonstrate a feasible pathway for incorporating these
useful features into the Nitime package.

\begin{figure}[h!]
  \includegraphics[width=.47\textwidth]{nitime.png}
  \caption{\label{centfig} Illustration of the highly comparative approach to time-series data from neuroscience.}
\end{figure}

\section{Results}\label{results}

I successfully implemented a handful of basic time-series analysis
functions from Matlab into python using partials, with basic support for
the Nitime data format. This proof-of-principle is
\href{https://github.com/benfulcher/hctsa_python}{here}.

\section{Conclusions}\label{conclusions}

Our results demonstrate that time-series analysis methods, discovered
using the \href{https://github.com/benfulcher/hctsa}{hctsa package}, can
be implemented natively in python in a systematic way, with support for
the time-series format used in Nitime. This will allow future work on
time-series analysis to be incorporated straightforwardly into an open
source environment.

%%%%%%%%%%%%%%%%%%%%%%%%%%%%%%%%%%%%%%%%%%%%%%
%%                                          %%
%% Backmatter begins here                   %%
%%                                          %%
%%%%%%%%%%%%%%%%%%%%%%%%%%%%%%%%%%%%%%%%%%%%%%

\begin{backmatter}

\section*{Availability of Supporting Data}
More information about this project can be found at: \url{https://github.com/benfulcher/hctsa\_python}. Further data and files supporting this project are hosted in the \emph{GigaScience} repository REFXXX.

\section*{Competing interests}
None

\section*{Author's contributions}
BF wrote the software and the report.

\section*{Acknowledgements}
The authors would like to thank the organizers and attendees of the 2015
OHBM Hackathon.

  
  
%%%%%%%%%%%%%%%%%%%%%%%%%%%%%%%%%%%%%%%%%%%%%%%%%%%%%%%%%%%%%
%%                  The Bibliography                       %%
%%                                                         %%
%%  Bmc_mathpys.bst  will be used to                       %%
%%  create a .BBL file for submission.                     %%
%%  After submission of the .TEX file,                     %%
%%  you will be prompted to submit your .BBL file.         %%
%%                                                         %%
%%                                                         %%
%%  Note that the displayed Bibliography will not          %%
%%  necessarily be rendered by Latex exactly as specified  %%
%%  in the online Instructions for Authors.                %%
%%                                                         %%
%%%%%%%%%%%%%%%%%%%%%%%%%%%%%%%%%%%%%%%%%%%%%%%%%%%%%%%%%%%%%

% if your bibliography is in bibtex format, use those commands:
\bibliographystyle{bmc-mathphys} % Style BST file
\bibliography{brainhack-report} % Bibliography file (usually '*.bib' )

\end{backmatter}
\end{document}
